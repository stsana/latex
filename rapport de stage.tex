\documentclass[12pt,a4paper]{scrbook}

\usepackage[utf8]{inputenc} %on peut remplacer utf8 par latin1
\usepackage[T1]{fontenc} % option T1 pour les accents
% \usepackage[french]{babel} %L’option french permet de respecter toutes les subtilités de la typographie française
\usepackage{textcomp} %certains caractères supplémentaires
\usepackage{amsmath,amssymb} %pour rédiger des formules mathématiques
\usepackage{lmodern} %choix d’une police de caractères
\usepackage{graphicx} %inclure des images
\usepackage[dvipsnames,svgnames]{xcolor} %utiliser les couleurs
\usepackage{microtype} %améliorations typographiques
\usepackage{listings} % pour pouvoir insérer du code R
\usepackage{lstautogobble}
\usepackage{lipsum}
\usepackage{tikz}
\usepackage{tikz-qtree}
\usepackage{hyperref}\hypersetup{pdfstartview=XYZ} %permet de gérer les liens, construit les bookmarks du pdf et rend la table des matières interactive (toujours à mettre en dernier) et ppuis le zoom par defaut
\lstset{language=R,
	breaklines=true,
	autogobble=true,
	backgroundcolor=\color{white},
    commentstyle=\color{red},   
    stringstyle=\color{green},   
	basicstyle=\ttfamily,
	showstringspaces=false,
	inputencoding=latin1,
	frame=single,
	stringstyle=\color{DarkGreen},
	keywordstyle=\color{DarkRed}\textbf,
	commentstyle=\textit,
}
%opening

\title{
	\begin{figure}
			\includegraphics[width=60mm]{logo-essfar}
			\hspace{185pt}
			\includegraphics[width=30mm]{logo-R}
	\end{figure}
	\centering
	\text{\Huge Rapport de stage}
\rule{\textwidth}{2pt} 
\texttt{\Large Vers la création d'un Package R pour l'estimation du temps de séjour d'un individu dans un stade de développement donné en présence d'une censure post-évènement d'intérêt}
\rule{\textwidth}{2pt}}
\author{\textbf{Rédigé par:} Sinclair Tsana\\
		\textbf{Sous l'encadrement de:} Dr. Patrice Takam Soh\\
		\\
		\text{Stage de 3e Année}}
		
\date{Septembre 2021}


\begin{document}
	
	\maketitle
	\tableofcontents
	\chapter*{Remerciements}
	Mes remerciements s'adressent en premier lieu à Monsieur le Directeur Général de l'ESSFAR, le Docteur \textbf{Patrick Seumen Tonou}, ainsi que le Directeur des Affaires Académiques, le Professeur \textbf{Etienne Tsam0} qui ont donné leurs accords pour tout mis en œuvre pour le bon déroulement de ce projet.\\
	
	Je tiens a remercier mon maître de stage \textbf{M. Patrice Takam Soh} pour son accompagnement et ses enseignements; ainsi que tout le corps enseignant et administratif de \textbf{l'ESSFAR} pour leur encadrement.\\
	
	Je remercie maintenant tous ceux ayant contribué de près ou de loin à ce projet.
		
	\chapter{Introduction}
	\par Le présent rapport de stage à été effectue dans le cadre de ma 3e année de formation en Mathématique et Informatique Appliquées a l'Économie à l'École Supérieure des Sciences de la Finance, de l'Assurance et des Risques(\textbf{ESFFAR}).
	Ce projet vise l'application des notions mathématiques vue en cours aux problèmes du monde réel et l'approfondissement de mes compétences en programmation R tout en contribuant à l'amélioration de cet outil fabuleux qu'est le logiciel R.
	\par En effet, logiciel R occupe une place de choix dans le domaine de la recherche en général et celui de la Science des données en particulier. Il est à la fois un langage de programmation et un logiciel destinés aux professionnels de la donnée. Dans le cadre de ma formation à l'ESSFAR la maîtrise de cet outil est très importante non seulement pour les simulations pratiques mais aussi pour l'insertion professionnelle.
	\par R est un logiciel libre de droit qui propose des nombreuses fonctionnalités. Il est néanmoins possible d'installer des suppléments, appelés packages, afin de les étendre. Le but de ce projet est donc de créer un de ces packages pour l’estimation du temps de séjour d’un individu dans un stade de développement donné.
	\chapter{Contexte}
		\section{Cadre de travail}
		Le stage s'est déroulé au sein du campus de l'ESSFAR dans la salle de cours des Licence 3 entre la période allant du \textbf{5 Juillet au 27 Septembre 2021}. Nous étions trois étudiants, chacun travaillant individuellement sur son projet. Nous n'hésitions pas a solliciter l'aide d'un camarade pour réfléchir ensemble en cas de blocage. Nous avions des rencontres avec l'enseignant deux ou trois fois par semaine pour faire le point et exposer nos difficultés à travers des présentations vidéo-projetés. Une connexion internet a été mise a notre disposition pour faciliter nos recherches. 
		\section{Objectifs du projet}
		Le Projet vise:
		\begin{itemize}
			\item L'application des méthodes mathématiques vues en cours au problèmes du monde réel.
			\item L'amélioration des compétences en programmation R
			\item L'initiation a la recherche reproductible en proposant un package R à la communauté scientifique.
		\end{itemize}
	Pour atteindre ces objectifs, nous avons organisé notre travail de la manière suivante:\\
	Le Chapitre 1 est une introduction générale du projet.Le chapitre 2 présente le contexte, le cadre de travail et les objectifs visés par le projet. Le chapitre 3 est une brève présentation du logiciel R et des outils de base nécessaires à la bonne marche du projet. Le chapitre 4 est une description détaille du travail effectué notamment des programmes écrits pour la réalisation du package. Le chapitre 5 présente les différents algorithmes d'estimation des paramètres par la méthode du maximum de vraisemblance. Le chapitre 6 est une conclusion générale du travail réalisé. 
	\chapter{Prérequis}
		\section{Programmation R}
			\subsection{Présentation du logiciel}
Le logiciel \textbf{R} \footnote{\url{http://cran.r-project.org/}} un outil très puissant de statistique crée par \textbf{Ross Ihaka} et \textbf{Robert Gentleman}. Il est a la fois un langage de programmation et un environnement de travail. Il fonctionne sous Linux, Windows et Mac. C'est aussi un logiciel gratuit et a code source ouvert (open-source).Tout le monde peut d'ailleurs contribuer a son amélioration en y intégrant de nouvelles fonctionnalités ou méthodes d'analyse non encore implémentes. C'est justement dans une logique de contribution que s'inscrit ce projet.\\
L'interface de développement \textbf{RStudio} est généralement associé a R pour augmenter la productivité des programmeurs. C'est un environnement de développement gratuit, libre et multi-plateforme pour R.
			\subsection{Les bases du logiciel}
R est un langage interprété et non compilé, c’est-à-dire que les commandes tapées au clavier sont directement exécutées sans qu’il soit besoin de construire un programme complet. Ensuite, la syntaxe de R est très simple et intuitive. Voici par exemple quelques opérations basiques.

	\begin{lstlisting}[language=R]
	> x <- 10  # Affectation
	> x
	[1] 10
	> vec <- 2:16  # Vecteur d'entiers entre 2 et 16
	> vec
	[1]  2  3  4  5  6  7  8  9 10 11 12 13 14 15 16
	> length(x)    # longuer du vecteur x
	[1] 15
	>seq(from=O,to=1,by=O.1) 
	> A <- matrix(1:9,nrow=3,ncol=3,byrow=TRUE)
	> A
	[,1] [,2] [,3]
	[1,]    1    2    3
	[2,]    4    5    6
	[3,]    7    8    9
	> maListe <- list(TRUE,1:5,c(1+2i,3),"chaine de caractères")
	> maListe
	[[1]]
	[1] TRUE		
	
	[[2]]				
	[1] 1 2 3 4 5			
	
	[[3]]
	[1] 1+2i 3+0i
				
	[[4]]
	[1] "chaine de caracteres"
	>maListe[3] # retrouver le 3e element de maListe
	>install.packages("stats")  # installer le package stats
	>library(dplyr)  # charger le package dplyr
			\end{lstlisting}
			
			\subsection{Procédure de Création d'un package}
Un package R est constitué d'un ensemble de fichiers et de répertoires dont la nomenclature est normalisée selon les directives publiées par le \textbf{CRAN}. Le répertoire racine doit porter le nom du package. Voici l'arborescence d'un package R:
			\begin{itemize}
				\item un répertoire \textbf{/R/}, qui contient les scripts R
				\item un répertoire \textbf{/man/}, qui contient les fichiers d'aide(chaque fonction publique doit être référencée dans un fichier d'aide).
				\item un répertoire \textbf{/data/}, qui contient les jeux de données 
				\item un répertoire \textbf{/tests/}, qui contient des tests bloquants qui arrêtent la compilation du package en cas d'erreur lors de leur exécution.
				\item un répertoire \textbf{/demo/}, qui contient des démonstration exécutées automatiquement lorsque l'utilisateur lance la commande demo().
				\item un répertoire \textbf{/inst/doc/}, qui contient des documentations utilisateurs, aussi appelées vignettes : ces documentations ont été réalisées avec \textbf{Markdown} et compilé avec le package \textcolor{red}{knitr} 
				\item un fichier \textbf{NAMESPACE}, qui précise la visibilité des fonctions,
				\item un fichier \textbf{DESCRIPTION}, qui décrit les différentes fonctions et leur utilisation
				\item un fichier \textbf{NEWS}, qui retrace l'historique des versions du package
				\item un fichier \textbf{TODO}, qui répertorie les fonctionnalités à implémenter ou les problèmes à corriger
			\end{itemize}
		\section{Calcul Intégral avec le logiciel R}
L'idée fondamentale de l'intégration numérique est d'approcher $I=\int_a^b f(x) dx$ dans le cas où on ne connais pas une primitive de f. Il y a deux façons de présenter le calcul:
\begin{itemize}
\item On approche I par une quadrature globale, c'est à dire qu'on remplace f par un polynôme d'interpolation.
\item On approche f par des polynômes par morceaux. Encore appelée méthode composite, cette méthode est la plus utilisée notamment dans les logiciels comme \textbf{Matlab}. Le logiciel R propose plusieurs packages et fonctions pour calculer ou estimer la valeur d'une intégrale simple dans un domaine fini ou infini. Le package \textbf{stats} propose par exemple la commande \textbf{integrate()}.
\end{itemize}
	\subsection{Intégrales simples}
Dans le cadre de ce travail, nous utiliserons les méthodes du triangle et des trapèzes pour approcher nos intégrales. Le programme s'écrit de la manière suivante:
	\begin{lstlisting}[language=R]
			integ1= function(f,a,b, ...,nrep=10){
				if(a>b) print('a doit etre inferieure a b')
				if(a == b) return(0)
				if(a<b){
					h = (b-a)/nrep; sub = seq(a,b,by=h); inter = sub[-c(1,length(sub))] 
					return( h*(sum(f(inter, ...)) + (1/2)*(f(sub[1], ...)+f(sub[length(sub)], ...))))
				}
			}
	\end{lstlisting}
	\subsection{Intégrales doubles}
Pour calculer une intégrale double, nous utilisons l'approximation d'une intégrale dans un triangle.
	\begin{lstlisting}[language=R]
	integrale.double.triangle = function(f,somet1,somet2,somet3,...){ 
	g1=0.4459; g2=0.0916; w1=0.22338; w2=0.10995    
	finter = function(g1,g2,g3){
	x = somet1[1]*g1 + somet2[1]*g2 + somet3[1]*g3
	y = somet1[2]*g1 + somet2[2]*g2 + somet3[2]*g3
	res = f(x,y,...)
	res}
	res1 = w1*( finter(g1,g1,1-2*g1)+finter(g1,1-2*g1,g1)+finter(1-2*g1,g1,g1) )
	res2 = w2*( finter(g2,g2,1-2*g2)+finter(g2,1-2*g2,g2)+finter(1-2*g2,g2,g2) )
	aire.triangle = (somet2[1]-somet1[1])*(somet3[2]-somet1[2])/2
	resultat = aire.triangle*(res1+res2)
	return(resultat)
	}
	\end{lstlisting} 
	R propose le package \textbf{cubature} et les fonctions \textbf{hcubature()} et \textbf{pcubature()} Ainsi que les packages \textbf{rmutil} et \textbf{pracma} pour approcher une intégrale. Cependant, pour de raison de vitesse, nous utiliserons l'algorithme du triangle suivant.
	\begin{lstlisting}[language=R]
	integ2 = function(f,lower,upper,...){ 
	somet1 = c(lower[1],lower[2]); somet2 = c(upper[1],lower[2]); 
	somet3 = c(upper[1],upper[2]);somet4 = c(lower[1],upper[2])
	res1 =  integrale.double.triangle(f,somet1,somet2,somet4,...)
	res2 =  integrale.double.triangle(f,somet3,somet4,somet2,...)
	res = res1 + res2
	return(res)
		    	}
    \end{lstlisting}
	\chapter{Description des programmes}
		\section{Importation des données}
159 cherelles ont été tirées au hasard sur 15 arbres dans une parcelle paysanne de Mban-
komo (à Yaoundé, Cameroun) avec récolte sanitaire toutes les semaines, en 1999. Ces fruits ont été choisis sur l’ensemble des niveaux des arbres (premier niveau de 0m à 0.5m, deuxième niveau de 0.5m à 1.5m, troisième niveau > 1.5m). On a noté chaque lundi matin, durant 20 semaines à partir du 26 avril 1999, le stade de développement de chacun de ces 159 fruits (1 = cherelle, 2 = jeune cabosse, 3 = cabosse adulte, 4 = cabosse mûre saine),
ainsi que leur état sanitaire (0 = fruit sain, 1 = fruit présentant au moins un sympôme de la pourriture brune, 2= fruit présentant au moins un symptôme de wilt, 3 = fruit rongé). Toutes ces informations ont été regroupés en 03 jeux de données(resp. data1=Etats sanitaire, data2=stade de développement et data3=données climatiques)qui nous ont permis de construire et de tester nos différents modèles. le fichier 
	\begin{lstlisting}[language=R]
	data1 <- read.table(dataEtats, header = TRUE, sep = "")
	data2 <- read.table(DataStadeDevelopement, header = TRUE, sep = "")		
	data3 <- read.table(Donnee_Climat_V1, header = TRUE, sep = "")	
	data.e <- data.frame(data1)[,-1]  # Suppression de la première colonne		
	data.s <- data.frame(data2)[,-1]
	data.cl <- data.frame(data3)[,-1]
	plv <- data.cl[,4]                # pluviométrie
	Ta <- (data.cl[,5]+data.cl[,6])/2 # température ambiante
	Tc <- (data.cl[,7]+data.cl[,8])/2 # température sous cacaoyère
	plv[plv==0]=0.01
	cl = data.frame(plv,Ta,Tc)
	\end{lstlisting}
	\section{Fonction qui donne les numéros des individus par stade}
	Avant toute chose, nous devons identifier nos individus selon leurs caractéristiques dans chaque stade à travers leurs indices.
	\begin{lstlisting}[language=R]
		Ik = function(k, datas){
			res=c()
			for(i in 1:nrow(datas)){
				y = datas[i,]
				if((sum(y==k)!=0) & (sum(y==k)!=1)) res[i]=i
				else res[i]=0
			}
			res
		}
	\end{lstlisting}
	\section{Statistique du nombre d'individus par stade}
Cette fonction permet de déterminer les indices de tous les individus ayant effectivement atteint chaque stade. 
	\begin{lstlisting}[language=R]
		nb1 = length(Ik(1,data.s)[Ik(1,data.s) !=0])
		nb2 = length(Ik(2,data.s)[Ik(2,data.s) !=0])
		nb3 = length(Ik(3,data.s)[Ik(3,data.s) !=0])
	\end{lstlisting}
	\section{Durée maximale discrète par stade}
La Durée maximale discrète par stade permet d'estimer le temps de séjour de chaque individu dans un stade k.
	\begin{lstlisting}[language=R]
		duree.disc = function(k,datas,L=7){
			ik = Ik(k, datas); ik1=ik[ik!=0];res=c()
			for(i in 1:length(ik1)){
				y=datas[ik1[i],]
				res[i] = (sum(y==k))*L + 1
			}
			res
		}
	\end{lstlisting}
	\section{Vecteur des poids}
Afin de rendre le concept plus réaliste nous avons définit un vecteur w0\_k qui détermine le poids de chaque facteur. k étant le stade courant, \textbf{datas} le jeu de données sur le statut sanitaire de l'ensemble des individus. Cette fonction renvoi un vecteur comportant les poids.\\
	\begin{lstlisting}[language=R,]
			w0k<- function(datas){
				w01 = 1/max(duree.disc(1,datas))
				w02 = 1/max(duree.disc(2,datas))
				w03 = 1/max(duree.disc(3,datas))
				res = c(w01,w02,w03)
				res
			}
	\end{lstlisting}
	\section{Vecteurs des covariables}
Considérons un individu qui entre au stade k a une date donné u, connaissant le vecteurs des variables qui agissent sur cet individu dans ce stade jusqu'à une date t (t>u), nous pouvons déterminer l'influence de chacun de ces facteurs à travers la fonction \textbf{x.tild} dont le code R est le suivant:\\
	\begin{lstlisting}[language=R]
		x_tild <- function(data.cl.cur,datas=data.s,u,v,k){#data.cl.cur=Ta,Tc ou plv
		u <- ceiling(u) 
		v <- floor(v)   
		n = v-u+1
		if(u>v){stop("not defined for u>v")}
		if(v==u)
			return(0)
		else
			res <- sum(data.cl.cur[1:(n-1)]) + (1/2)*(data.cl.cur[u]+data.cl.cur[v])
		return(res*w0k(datas)[k])
		}
		\end{lstlisting}

		\textbf{Description} \\
La fonction \textbf{x.tild()} détermine la part de chaque facteur qui influence la croissance d'un individu i dans un stade donné k.La fonction x\_tild renvoie une valeur entière qui estime l'influence du climat courant sur le développement de l'individu.\\
		\\
		\textbf{Arguments}\\
		\begin{tabular}{cc}
			data.cl.cur  & climat courant(plv, Ta ou Tc)\\
			u & Date d'entrée au stade \\
			v & Dernier jour au stade \\
			tp & dernier jour de l'étude \\
			k & stade courant \\
			\\
		\end{tabular} 
		
		\section{Fonction de Hazard Cumulée}
Nous pouvons donc être intéressé par l'estimation de l'incidence cumulé de chacun des facteurs dans le stade k. Pour se faire, nous avons définit la fonction Cumulative Hazard (Hk).
		\begin{lstlisting}[language=R]
		alpha.1 = rep(1,3) #Initialisation des paramètres 
		mu.1 = rep(1,3)
		
		Hk <- function(v,u,climat=cl,datas=data.s,alpha=alpha.1,mu=mu.1,k=1){
			if(v<u){
				stop("not defined for v<u")}
			else if(v==u)
			return(0)
			else{
				plv = climat[,1];Ta=climat[,2];Tc=climat[,3]
				v.cum_clim <- c(x_tild(plv,datas,u,v,k),
				x_tild(Ta,datas,u,v,k),
				x_tild(Tc,datas,u,v,k))
				som <- sum(alpha* (v.cum_clim^mu))
				return(som)
			}
		}
		\end{lstlisting}
		\section{Fonction de Survie}
Lorsqu'on se place dans un stade de développement donné k, l'évènement d'intérêt est le passage au stade suivant k+1. Nous pouvons estimer la probabilité que l'individu reste sain a la fin du stade k a l'aide de la fonction \textbf{Sk()} définie comme suit:\\
		\begin{lstlisting}[language=R]
			Sk <- function(v,u,climat=cl,datas=data.s,alpha=alpha.1,mu=mu.1,k=1){
				if(v>=u){
					return(exp(-Hk(v,u,climat,datas,alpha,mu,k)))
				}else{
					stop("Verifiez que v>=u")
				}
			}
		\end{lstlisting}
		\section{Fonction densité de la durée de vie au stade k}
Ici, nous voulons estimer la loi d'une variable aléatoire $T_k$ qui désigne le temps passé par un individus sain au stade k.La variable aléatoire $T_k$ est donc non-négative et nous supposons qu'elle est indépendamment et identiquement distribué de densité $\psi\theta_k$.\\
		\begin{lstlisting}[language=R]
phi_k <- function(v,u,climat=cl,datas=data.s,alpha=alpha.1,mu=mu.1,k=1){
	Ta.tild <- x_tild(Ta,datas,u,v,k)
	Tc.tild <- x_tild(Tc,datas,u,v,k)
	plv.tild <- x_tild(plv,datas,u,v,k)
	v.cum_clim <- c(plv.tild,Ta.tild,Tc.tild)
	v.inter <- c(plv[v],Ta[v], Tc[v])
	betha <- alpha*mu*v.inter
	s <- sum(betha*v.cum_clim^(mu-1))
return(Sk(v,u,climat,datas,alpha,mu,k)*s)
}
		\end{lstlisting}	
		\section{Statut sanitaire}
Pour un individu i ayant été observe au moins une fois au stade k, on distingue 4 possibilités:
			\begin{itemize}
				\item Soit l'individu i reste sain jusqu'au dernier jour d'observation
				\item Soit il est sain jusqu'au premier jour au stade k+1
				\item Soit il est attaqué au stade k
				\item Soit il est attaqué pendant la transition du stade k au stade k+1.
			\end{itemize}
		\begin{lstlisting}[language=R]
			statut <- function(i,k=1,tp=20,datas=data.s, datae=data.e){
				ik = Ik(k,datas)
				if(ik[i]==0){stop("L'individu n'est pas dans ce stade")}
				else{
					y = datas[i,]
					eik <- min((1:length(y))[y==k]) # Premier jour de i au stade k
					eikp1 <- max((1:length(y))[y==k]) # Dernier jour de i au stade k
					if(eikp1<tp){
						#Cas ou i a changé de stade
						if((datae[i,eikp1]==0) & (datas[i,eikp1] != datas[i,eikp1+1]) & (datae[i,eikp1+1]==0))
						res=1
						if((datae[i,eikp1]==0) & (datas[i,eikp1] != datas[i,eikp1+1]) &(datae[i,eikp1+1]!=0))
						res=2
						# Cas ou i n'a pas changé de stade
						if(datae[i,eikp1]!=0)
						res=3
					}
					if(eikp1==tp){
						#Ici l'individu n'a pas pu changé car il est observé au stade k à tp
						# Cas ou i est observé sain a tp
						if(datae[i,eikp1]==0) res=4
						# Cas ou i est observé attaqué à tp
						if(datae[i,eikp1]!=0) res=5
					}
				}
				res
			}
			
		\end{lstlisting}

		\section{Statut de tous les individus dans le stade k}
Ici nous définissons une fonction qui prends en entrée le numéro d'un stade et donne le statut de tous les individus dans le stade en question en vue de les classer et les différencier dans l'étude.
		\begin{lstlisting}[language=R]
		statut.data.k = function(k=1,tp=20,datas=data.s, datae=data.e){
			res=c()
			ik = Ik(k,datas);ik=ik[ik!=0]
			for(i in ik)
			res[i] = statut(i,k,tp,datas, datae)
			res
		}
		\end{lstlisting}
	\chapter{Estimation des paramètres}
	\section{Estimation des paramètres au stade 1}
		\subsection{Calcul de la probabilité d'être observé sain au stade suivant}
	\begin{lstlisting}[language=R]
  proba1 = function(i,climat=cl,datas=data.s,alpha=alpha.1,mu=mu.1,k=1,L=7,w0k=w0.1){
  y = datas[i,]
  eik <- min((1:length(y))[y==k]) 
  eikp1 <- max((1:length(y))[y==k]) 
  eik.c = eik*L + 1; eikp1.c = eikp1*L + 1;v0.1=eikp1.c-L;v0.2=eikp1.c
  f = function(u) return( Sk(u,v=v0.1,climat=climat,alpha=alpha,mu=mu,k=k,w0k=w0k) )    
  g = function(u) return( Sk(u,v=v0.2,climat=climat,alpha=alpha,mu=mu,k=k,w0k=w0k) )
  f = Vectorize(f); g= Vectorize(g)`
  res1 = integ1(f,eik.c-L,eik.c); res2 = integ1(g,eik.c-L,eik.c)
  resf = (1/L)*(res1-res2)
  resf
}
	\end{lstlisting}
	\subsection{Calcul de la probabilité d'être observé infecté au stade suivant}
	\begin{lstlisting}[language=R]
	proba2 = function(i,climat=cl,datas=data.s,alpha=alpha.1,mu=mu.1,k=1,L=7,w0k=w0.1){
  y = datas[i,]
  eik <- min((1:length(y))[y==k]) 
  eikp1 <- max((1:length(y))[y==k]) 
  eik.c = eik*L + 1; eikp1.c = eikp1*L + 1;v0.1=eikp1.c-L;v0.2=eikp1.c
  f = function(u) return( Sk(u,v=v0.1,climat=climat,alpha=alpha,mu=mu,k=k,w0k=w0k) ); f = Vectorize(f)
  res1=(1/L)*integ1(f,eik.c-L,eik.c)
  res2 = (1/L^2)*integ2(Sk,c(eik.c-L,eikp1.c-L),c(eik.c,eikp1.c),climat,alpha,mu,k,w0k)
  res = res1-res2; res
}
    \end{lstlisting}

	\subsection{Calcul de la probabilité d'être observé infecté au dernier jour d'observation(avant la fin de l'étude)}
	\begin{lstlisting}[language=R]
proba3 = function(i,climat=cl,datas=data.s,alpha=alpha.1,mu=mu.1,k=1,L=7,w0k=w0.1){
  y = datas[i,]
  eik <- min((1:length(y))[y==k]) 
  eikp1 <- max((1:length(y))[y==k]) 
  eik.c = eik*L + 1; eikp1.c = eikp1*L + 1;
  res = (1/L^2)*integ2(Sk,c(eik.c-L,eikp1.c-L),c(eik.c,eikp1.c),climat,alpha,mu,k,w0k)
  res
}
	\end{lstlisting}

	\subsection{Calcul de la probabilité d'être observé sain au dernier jour d'observation de l'étude}
	\begin{lstlisting}[language=R]
proba4 = function(i,climat=cl,datas=data.s,alpha=alpha.1,mu=mu.1,k=1,L=7,tp=20,w0k=w0.1){
  y = datas[i,]
  eik <- min((1:length(y))[y==k]) 
  eikp1 <- tp 
  eik.c = eik*L + 1; eikp1.c = eikp1*L + 1; v0.2=eikp1.c
  f = function(u) return( Sk(u,v=v0.2,climat=climat,alpha=alpha,mu=mu,k=k,w0k=w0k) ); f = Vectorize(f)
  res=(1/L)*integ1(f,eik.c-L,eik.c)
  res 
}
	\end{lstlisting}

	\subsection{Calcul de la probabilité d'être observé attaqué au dernier jour d'observation de l'étude}
	\begin{lstlisting}[language=R]
	proba5 = function(i,climat=cl,datas=data.s,alpha=alpha.1,mu=mu.1,k=1,L=7,tp=20,w0k=w0.1){
  y = datas[i,]
  eik <- min((1:length(y))[y==k]) 
  eikp1 <- max((1:length(y))[y==k]) 
  eik.c = eik*L + 1; eikp1.c = eikp1*L + 1;v0.1=eikp1.c-L;v0.2=eikp1.c
  res = integ2(Sk,c(tp*L+1-L,eik.c-L),c(eik.c,eikp1.c),climat,alpha,mu,k,w0k)
}
	\end{lstlisting}
	
	\subsection{Calcul de la vraisemblance au stade 1}
	On rappel qu'on ne considérera que les individus ayant été observé au moins deux fois dans le stade.
	\begin{lstlisting}[language=R]
statut.stade1 = statut.data.k(1)
Lvrais1 = function(coefs,statut=statut.stade1,climat=cl,datas=data.s,k=1,L=7,tp=20,w0k=w0.1){
  alpha = coefs[1:3];  alpha = exp(alpha/100)
  mu = coefs[4:6]; mu=exp(-mu/100)
  ik=Ik(k,data.s);ik1=ik[ik!=0];res=c()
  for(i in ik1){
    if(statut[i]==1) res[i]=proba1(i,climat,datas,alpha,mu,k,L,w0k)
    if(statut[i]==2) res[i]=proba2(i,climat,datas,alpha,mu,k,L,w0k)
    if(statut[i]==3) res[i]=proba3(i,climat,datas,alpha,mu,k,L,w0k)
    if(statut[i]==4) res[i]=proba4(i,climat,datas,alpha,mu,k,L,tp,w0k)
    if(statut[i]==5) res[i]=proba5(i,climat,datas,alpha,mu,k,L,tp,w0k)
  }
  res=res[!is.na(res)]; res[res==0]=0.0001
  resf=-sum(log(res)) 
  resf
}
    \end{lstlisting}
	
	\chapter{Conclusion}	
	Je tire un bilan très positif de ce stage, qui fut une expérience très enrichissante tant sur le plan académique et personnel. Sur le plan académique d’abord, j’ai pu implémenter les notions vues au cours de l'année académique comme l'estimation Paramétrique et la théorie des tests. J’ai rempli les objectifs fixés, à savoir améliorer mes compétences en programmation R ainsi que l'application des notions mathématiques vue en cours aux problèmes du monde réel. Sur le plan personnel ensuite, j'ai compris l'importance de la patience dans les études scientifiques et surtout en programmation. Ainsi, j'ai appris la procédure a suivre lorsque je sèche devant un problème. Je me souviens encore lorsque mon  encadreur me disait que: "Je veut que tu apprenne à réfléchir comme ça". Au cours de cette période, comme dans toute phase d’apprentissage, il m’est par ailleurs arrivé de faire quelques erreurs que j’ai pu rapidement corriger en me rapprochant de mon encadreur et en respectant scrupuleusement ses consignes. Grâce aux acquis d’une méthodologie de travail forte que mon encadreur m’a transmise,combinée à la formation théorique que j’ai reçu, je suis aujourd’hui apte à animer un projet de statistique sous le logiciel R. Finalement au sortir de cette expérience, j'ai renforcer ma passion pour l'analyse des données massives qui sera certainement l'objet de mes prochains travaux.
	
	\begin{thebibliography}{1}
		\bibitem{texbook}
			Patrice Takam Soh \textit{MCEM-Algorithm for estimating lifetime distribution in
presence of interval censured observations and a censoring
variable. Application to cocoa data}.
		\bibitem{texbook}
			https://cran.r-project.org/web/views/Optimization:	Stefan Theussl, Florian Schwendinger, Hans W. Borchers
		\bibitem{texbook}
			CRAN Task View: \textit{Optimization and Mathematical Programming}
		\bibitem{texbook}
			Paulo Cortez \textit{Modern Optimization with R}
		\bibitem{texbook}
			https://rdrr.io: \textit{Practical Numerical Math Functions}
		\bibitem{texbook}
			https://stackoverflow.com/
		\bibitem{texbook}
			Donald E. Knuth (1986) \emph{The \TeX{} Book}, Addison-Wesley Professional.
	\end{thebibliography}
\end{document}
